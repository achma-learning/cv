\let\nofiles\relax
\documentclass[margin,line]{res}
\usepackage[colorlinks=true, urlcolor=blue]{hyperref}
\usepackage[utf8]{inputenc}
\usepackage[T1]{fontenc}
\usepackage{microtype}
\usepackage{amsmath,amssymb}
\usepackage{lastpage}
\usepackage{fancyhdr}
\usepackage{etaremune}

\oddsidemargin -.5in
\evensidemargin -.5in
\voffset -60pt
%\topmargin -.2in
\headsep 0pt
\textwidth=6.0in
\textheight=12in
\itemsep=0in
\parsep=0in

% Headings
\pagestyle{fancy}
\lhead{Roman Ring --- Curriculum Vitae}
\chead{}
\rhead{\thepage\ of \pageref*{LastPage}}
\lfoot{}
\cfoot{}
\rfoot{}
\renewcommand{\headrulewidth}{0.4pt}
%\renewcommand{\footrulewidth}{0.4pt}

\begin{document}

\newcommand{\myname}{Roman Ring --- Curriculum Vitae}
\newlength{\mynamewidth}
\settowidth{\mynamewidth}{\namefont\myname}

\name{\hspace*{0.5\textwidth}\hspace{-0.5\mynamewidth} \myname \vspace*{.1in}}
% On the first page, have no header.
\thispagestyle{empty}

\begin{resume}

\section{\sc Contact Information}
\hfill {inoryy@gmail.com}\\

\section{\sc Education}
{\bf M.S., Computer Science,} University of Tartu, Estonia \hfill {\it September 2018---July 2020}\vspace*{+.05in}\\
{\bf B.S., Mathematical Statistics,} University of Tartu, Estonia \hfill {\it September 2014---July 2018}\\
\vspace*{-.1in}
\begin{itemize}
\vspace*{-.05in}
\item[ ] G.P.A.: 4.53/5.0
\vspace*{-.05in}
\item[ ] Thesis: \href{http://dspace.ut.ee/handle/10062/61039}{Replicating DeepMind StarCraft II RL Benchmark with Actor-Critic Methods}
%\vspace*{-.05in}
%\item[ ] Advisors: Ilya Kuzovkin, Tambet Matiisen
\end{itemize}

\section{\sc Employment}
{\bf Research Engineer Intern,} DeepMind (Google UK), London \hfill {\it June 2019---September 2019}\\
%\vspace*{-.1in}
%\begin{itemize}
%\item[ ] Assisting with research in the domain of deep reinforcement learning
%\end{itemize}
%\vspace*{-.05in}

{\bf Research Assistant,} Comput. Neuroscience Research Group \hfill {\it February 2018---Present}\\
\vspace*{-.1in}
\begin{itemize}
\item[ ] Assisting with research in the domain of deep reinforcement learning
\end{itemize}
\vspace*{-.05in}

{\bf Research Assistant,} PerkinElmer \hfill {\it June 2018---December 2018}\\
\vspace*{-.1in}
\begin{itemize}
\item[ ] Improving instance segmentation pipeline in fluorescent medical imaging
\end{itemize}
\vspace*{-.05in}

{\bf Senior Web Developer,} KNP Labs \hfill {\it September 2011---February 2015}\\
\vspace*{-.1in}
\begin{itemize}
\item[ ] Development and support of complex web based applications (banking, education, retail)
\item[ ] Coaching junior developers with hands-on workshops, pair programming sessions, PR reviews 
\end{itemize}
\vspace*{-.05in}
{\bf Web Developer,} Attitude OÜ \hfill {\it September 2010---September 2011}\\
\vspace*{-.1in}
\begin{itemize}
\item[ ] Development and support of web based applications
\end{itemize}

%{\bf Web Developer,} A2B Grupp OÜ \hfill {\it  June 2010---September 2010}\\
%{\bf Web Developer,} Las Flores Tours \hfill {\it  February 2010---June 2010}

\section{\sc Skills}
{\it Expert in:} Python, PHP, JavaScript; Keras, Symfony, Doctrine, Angular; git
\vspace*{+.01in}\\
{\it Proficient in:} C++, R, Java, HTML, CSS; Tensorflow, PyTorch, NumPy, SciPy; AWS
\vspace*{+.01in}\\
{\it Experience in:} Bash, MATLAB, SAS, LaTeX; Caffe, Theano, OpenCV; vim

%\section{\sc Projects}
%{\bf Starcraft II Reinforcement Learning Agent,} (Python, Tensorflow) \hfill {\it October 2017---Present}\\
%\vspace*{-.1in}
%\begin{itemize}
%  \item[ ] 
%\end{itemize}
%\vspace*{-.1in}
%{\bf Coders Strike Back AI Bot,} (C++) \hfill {\it December 2016---May 2017}\\
%\vspace*{-.1in}
%\begin{itemize}
%  \item[ ] 
%\end{itemize}

\section{\sc Open Source}
Symfony Web Framework, Doctrine ORM (contributor)\vspace*{+.05in}\\
TensorFlow, PySC2, SciPy, StatsModels (minor contributor)\vspace*{+.05in}\\
\href{https://github.com/Inoryy/reaver-pysc2}{Reaver: SC2 DRL Agent}, \href{https://github.com/Inoryy/csb-ai-starter}{CSB AI Starter}, \href{https://github.com/KnpLabs/mailjet-api-php}{Mailjet PHP API} (creator)

\section{\sc Talks}
Reinforcement Learning (Guest Lecture, University of Tartu) \hfill {\it December 2018}\vspace*{+.05in}\\
Deep Reinforcement Learning (DevClub, Tallinn) \hfill {\it June 2018}\vspace*{+.05in}\\
Behavior Driven Development with Behat and Mink (DevClub, Tallinn) \hfill {\it January 2013}

\section{\sc Teaching}
%Calculus I, TA (University of Tartu) \hfill {\it Autumn 2018}\vspace*{+.05in}\\
Neural Networks, TA (University of Tartu) \hfill {\it Spring 2019}\\
Deep Reinforcement Learning, TA (University of Tartu) \hfill {\it Autumn 2018}

\section{\sc Competitions}
Kaggle 2018 Data Science Bowl (277/3634, team) \hfill {\it April 2018}\vspace*{+.05in}\\
% Kaggle Recruit Restaurant Visitor Forecasting (233/2158) \hfill {\it February 2018}\vspace*{+.05in}\\
%Kaggle Carvana Image Masking Challenge (193/735) \hfill {\it September 2017}\vspace*{+.05in}\\
Codingame AI Contest Coders of the Caribbean (28/3623) \hfill {\it April 2017}\vspace*{+.05in}\\
%Kaggle State Farm Distracted Driver Detection (449/1440) \hfill {\it April 2016}\vspace*{+.05in}\\
Hackerrank University World Cup (22/4466, team) \hfill {\it September 2015}\vspace*{+.05in}\\
IEEEXtreme 8.0 (208/1853, team) \hfill {\it September 2014}

\section{\sc Awards}
Estonian National Contest for University Students, B.S. programme, 2nd prize \hfill {\it December 2018}\vspace*{+.05in}\\
DevClub Best Talk Award \hfill {\it December 2018}\vspace*{+.05in}\\
Cybernetica AS  Master's Fellowship \hfill {\it October 2018}

%\vspace*{-.1in}
%{\bf Certificates}
%\begin{itemize}
%    \item[ ] Zend Certified Engineer PHP 5.3 \hfill {\it November 2012}
%\end{itemize}

\section{\sc Relevant Coursework}
Information Theory, Stochastic Processes, Matrix Calculus, Monte-Carlo Methods, Machine Learning, Neural Networks, Data Analysis I-II, Numerical Analysis, Mathematical Analysis I-III, Probability Theory \& Statistics I-II, Algebra (Abstract \& Linear)
% Non-Parametric Statistics, Intro to Comp. Neuroscience, Machine Translation
\vspace*{-.1in}
\\\\
{\it Online:} Machine Learning (Stanford CS229), CNNs for Visual Recognition (Stanford CS231n), Deep Learning for NLP (Stanford CS224d),
Intro to AI (Berkeley CS188), DRL Bootcamp (Berkeley), Reinforcement Learning (UCL), Deep Reinforcement Learning (Berkeley CS294)

%\section{\sc Research Interests}
%Reinforcement Learning, specifically model-free methods. Particularily interested in working on improving sample efficiency and stability of RL algorithms by applying variance reduction techniques.

\end{resume}
\end{document}
